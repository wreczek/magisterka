%\chapter{Wprowadzenie}
%\label{cha:wprowadzenie}
%
%\LaTeX~jest systemem składu umożliwiającym tworzenie dowolnego typu dokumentów (w~szczególności naukowych i technicznych) o wysokiej jakości typograficznej (\cite{Dil00}, \cite{Lam92}). Wysoka jakość składu jest niezależna od rozmiaru dokumentu -- zaczynając od krótkich listów do bardzo grubych książek. \LaTeX~automatyzuje wiele prac związanych ze składaniem dokumentów np.: referencje, cytowania, generowanie spisów (treśli, rysunków, symboli itp.) itd.
%
%\LaTeX~jest zestawem instrukcji umożliwiających autorom skład i wydruk ich prac na najwyższym poziomie typograficznym. Do formatowania dokumentu \LaTeX~stosuje \TeX a (wymiawamy 'tech' -- greckie litery $\tau$, $\epsilon$, $\chi$). Korzystając z~systemu składu \LaTeX~mamy za zadanie przygotować jedynie tekst źródłowy, cały ciężar składania, formatowania dokumentu przejmuje na siebie system.
%
%%---------------------------------------------------------------------------
%
%\section{Cele pracy}
%\label{sec:celePracy}
%
%
%Celem poniższej pracy jest zapoznanie studentów z systemem \LaTeX~w zakresie umożliwiającym im samodzielne, profesjonalne złożenie pracy dyplomowej w systemie \LaTeX.
%
%\subsection{Jakiś tytuł}
%
%\subsubsection{Jakiś tytuł w subsubsection}
%
%
%\subsection{Jakiś tytuł 2}
%
%%---------------------------------------------------------------------------
%
%\section{Zawartość pracy}
%\label{sec:zawartosc_pracy}
%
%W rodziale~\ref{cha:pierwszyDokument} przedstawiono podstawowe informacje dotyczące struktury dokumentów w \LaTeX u. Alvis~\cite{Alvis2011} jest językiem 



% ######################## ROZDZIAŁ 1 ##########################
\section*{Streszczenie pracy}

Celem niniejszej pracy jest zbadanie skuteczności różnych algorytmów klasyfikacji i ich rozszerzeń w ocenie jakości odlewów. Pomocne w realizacji projektu może być opracowanie oprogramowania pozwalającego na kompleksowe przebadanie wszystkich zaimplementowanych algorytmów. Korzystając z tej aplikacji, zostaną przeprowadzone eksperymenty, w których przygotowane oprogramowanie zostanie przetestowane w różnych warunkach i konfiguracjach, a także dla różnych algorytmów oraz danych wejściowych, którymi są zdjęcia przekrojów odlewów (zdjęcia mikrostruktury) lub informacje na temat materiału (np. typ, skład). 

Praca ma charakter badawczy, gdyż jak wykazał przegląd literatury, prac naukowych na ten temat (tj. pod kątem wykorzystania uczenia maszynowego w celu oceny jakości odlewów) oraz źródeł wskazujących na praktyczne stosowanie oprogramowania o podobnym przeznaczeniu jest niewiele. Dlatego też możliwości wykorzystania uczenia maszynowego do oceny jakości odlewów będą przetestowane z użyciem najbardziej uniwersalnych metod, a także zostaną porównane te wyniki z rezultatami sieci neuronowych.


\section*{Abstract of master's thesis}
The aim of this research is to see how efficient various categorization algorithms and extensions are at determining casting quality. The development of software that allows for extensive testing of all developed algorithms could be beneficial to the project's implementation. Experiments will be conducted using this application, in which the developed software will be tested in a variety of situations and configurations, as well as for a variety of algorithms and input data, such as images of casting sections (pictures of microstructure) or material information (e.g., type, composition).

Because there are few scholarly articles on the issue (i.e., using machine learning to assess the quality of castings) and few sources suggesting the actual usage of software for comparable reasons, the work is of a research character. As a consequence, the capabilities of utilizing machine learning to assess the quality of castings will be explored using the most general approaches, with the results compared to neural networks.

















