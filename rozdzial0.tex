\section*{Streszczenie pracy}

Celem niniejszej pracy jest zbadanie skuteczności różnych algorytmów klasyfikacji i~ich rozszerzeń w~ocenie jakości odlewów. Pomocne w~realizacji projektu może być opracowanie oprogramowania, bądź skorzystanie z gotowego rozwiązania, pozwalającego na kompleksowe przebadanie wszystkich zaimplementowanych algorytmów. Korzystając z~tej aplikacji, zostaną przeprowadzone eksperymenty, w~których przygotowane algorytmy i metody uczenia maszynowego zostaną przetestowane w~różnych warunkach i~konfiguracjach, a~także dla różnych danych wejściowych, którymi są zdjęcia przekrojów odlewów (zdjęcia mikrostruktury) lub informacje na temat materiału (np. typ, skład). 

Praca ma charakter badawczy, gdyż jak wykazał przegląd literatury, prac naukowych na ten temat (tj. pod kątem wykorzystania uczenia maszynowego w~celu oceny jakości odlewów) oraz źródeł wskazujących na praktyczne stosowanie oprogramowania o~podobnym przeznaczeniu jest niewiele. Dlatego też skuteczność algorytmów uczenia maszynowego w ocenie jakości odlewów zostanie przetestowana z~użyciem najbardziej uniwersalnych metod. Wyniki uzyskane przez klasyczne metody uczenia maszynowego oraz przez sieci neuronowe zostaną ze sobą porównane, biorąc pod uwagę takie aspekty, jak interpretowalność rezultatów, łatwość implementacji modelu, prostotę algorytmu czy czas uczenia.


\section*{Abstract of master's thesis}

The aim of this research is to see how efficient various categorization algorithms and extensions are at determining casting quality. The development of software or the use of a ready-made solution that allows for extensive testing of all developed algorithms could be beneficial to the project's implementation. Experiments will be conducted using this application, in which the developed algorithms and machine learning methods will be tested in a~variety of situations and configurations, as well as for a~variety of input data, such as images of casting sections (pictures of microstructure) or material information (e.g., type, composition).

Because there are few scholarly articles on the issue (i.e., using machine learning to assess the quality of castings) and few sources suggesting the actual usage of software for comparable reasons, the work is of a~research character. As a~consequence, the capabilities of utilizing machine learning to assess the quality of castings will be explored using the most general approaches. The results obtained by classical machine learning methods and by neural networks will be compared with each other, taking into account aspects such as the interpretability of results, ease of model implementation, algorithm simplicity, and learning time. 















